\section{Method reflection}
\begin{itemize}
    \item Streaming API was the better option on the free tier. But this solution is not flexible enough to see how different topics get discussed on Twitter: the data collection ran over 2 months, and this might not be feasible. (braucht eigentlich nicht mehr gemacht werden)
    \item Accessibility Issues: too much reliance on colors alone. It would have been better to also include texture patterns. Both color and patterns are qualitative nominal variables, so they can encode the same information (\cite[1860]{bornerDataVisualizationLiteracy2019}).
    \item Using \emph{Shape Up} to plan the development of the tool worked well. It is possible, however, that the clear distinction pipeline---d3-visualizations---meant that some parts of the study, like the pipeline, took more time than necessary.
    \item The participants in the interview study were mostly students. This was partly because of the ongoing Covid-19 pandemic, which made it necessary to use online conferencing software to recruit the participants and conduct the interviews with them. Face-to-face interviews with a more diverse group of participants can yield further results. This should be considered for future studies.
    \item Using Observable sped up the development process and was the right choice for the prototype. For a final product, however, the limitations of Observable are too severe, especially when using databases. Notebooks with databases require permissions to be set up, and for this users need to create an Observable account and join the team workspace where the notebook is located. Instead of using Observable, users should be able to use a web tool without logging in. One example for a data analysis tool on the web is \emph{Blacklight}\footnote{https://themarkup.org/blacklight/}, where users can scan other web pages for trackers.
\end{itemize}
