\section{Results}
\begin{itemize}
    \item The findings fell into two categories: motivational factors and hindering factors.
\end{itemize}

\subsection{Sample description}
\begin{itemize}
    \item Six participants, aged 23-26, all students
    \item 3 male, 3 female
    \item they were chosen as participants because they have little to no knowledge in big data and data visualization, thus qualifying as \say{laymen}.
    \item 3 read Twitter daily, 1 person at least a few times a week. 1 Participant never reads contents on twitter itself.
    \item 4 people see twitter content at least weekly on other platforms like Instagram or Reddit, two rarely
    \item 1 person reports that they frequently interact with twitter contents. The other 5 rarely or never do this
    \item all of the participants reported that they rarely or never post tweets themselves
\end{itemize}

\subsection{Motivational factors}
Motivational factors were factors that motivated participants to use the dashboard, explore the dataset, and analyse the results. 10 motivational factors were identified in total which will be described in the following, along with anchor examples.

\subsubsection*{Interesting results}
Participants reported that they found the results they gathered from the visualizations interesting. One example is a participant who found it interesting that retweets make up most of the activity on twitter:

\begin{quote}
    Aber schon spannend eigentlich, dass alles... also viele hauptsächlich Retweets sind und nicht eigene Tweets. (m23, l. 18)
\end{quote}

Others found the impact of retweets on the overall sentiment interesting:

\begin{quote}
    Ich fands sehr interessant zu sehen, wie so Retweets die Stimmung beeinflussen. (w25, l. 80)
\end{quote}

Participants also expressed that they liked to have the ability to see how a certain topic was discussed by the German twittersphere:

\begin{quote}
    Aber ja, das ist schon interessant da mal nen Begriff reinzutippen und zu gucken, was passiert. (m24, l. 82)
\end{quote}

Some participants said that it was nice to see their own gut-feeling being backed by data:

\begin{quote}
    Ich finds sehr interessant zu sehen, wie so Retweets die Stimmung beeinflussen. Weil wenn man selbst Twitter benutzt merkt man halt auch selbst [...] dass mich persönlich das aufregt. Und das ist eigentlich sehr interessant gewesen zu sehen, dass das nicht nur mein Empfinden ist, sondern dass sich das generell in Twitter wiederfindet. Dass die Stimmung durch immer-wiederholen des gleichen Themas beeinflusst wird. (w25, l. 80)
\end{quote}

\subsubsection*{Filters can be applied quickly}
During testing, all participants used the toggles to show and hide retweets and neutral tweets multiple times in a row. This was likely done to compare the impact of retweets and neutral tweets on the data set.

\begin{quote}
    B: Und jetzt kann ich ja mal gucken wie sich das ändert, wenn ich die Retweets ausblende

    Screen: blendet die Retweets wiederholt aus und wieder ein (w26a, ll. 73-74)
\end{quote}

This behaviour happened both during the free exploration and the task-solving phase of the interview.

\subsubsection*{Easily understandable visualizations}
Especially the bar chart which showed the tweet volume per day seemed to be easy to read for participants. This led to a quicker understanding of the data set and made it easier for participants to verify their assumptions. This participant, for example, wanted to see if a real-life event she knew of was reflected in the data set during free exploration:

\begin{quote}
    B: Ich probier es mal mit "Stuttgart", denn da gab es ja glaub ich eine der ersten dieser Hygiene-Demos
    Screen: gelbe Balken werden zwischen dem 21. und dem 24. Juni sichtbar
    B: Ah ja interessant, dann war diese Demo bestimmt am 21. Juni. Interessant. (w26b, ll. 6-8)
\end{quote}

Another participant explicitely said that both visualizations were easily understandable, even without further explanation:

\begin{quote}
    Und die Beschriftung der Achsen war auch so eindeutig, dass man... wahrscheinlich hätte ich da nicht einmal den Infotext gebraucht und die Grafiken hätten gereicht. (m26, l. 83)
\end{quote}

\subsubsection*{Supporting explanatory texts}
Participants reported that the explanatory texts that accompanied the visualizations were helpful. They aided them both in the exploration phase:

\begin{quote}
    Ich hab mich gerade kurz gefragt, was neutrale Tweets sind, aber hier unten steht ja die Erklärung dazu. (m26, l. 22)
\end{quote}

as well as while solving the tasks:

\begin{quote}
    I: Wie leicht ist es dir gefallen, die Aufgaben zu beantworten?

    B: Also ich würde sagen ziemlich leicht [...] Mit den Erklärungen fand ich die schon gut zu lösen. (w26b, ll. 68-69)
\end{quote}

\subsubsection*{Encouraging analysis and own thinking}
Another point that participants liked was that the visualizations encouraged them to think about the meaning for themselves. One participant expressed joy that the interactive visualizations helped her answering questions she had on her own:

\begin{quote}
    Ne es gab ja so unendlich viele Corona-Visualisierungen immer. Und ich find das ganz cool, dass man hier sehr offen sich das anschauen kann. Weil sonst geben so Visualisierungen ja ziemlich genau vor, was man überhaupt sehen kann. (w26b, l. 63)
\end{quote}

Another participant liked that the tasks asked them to reflect on what they could read from the visualizations:

\begin{quote}
    Ja gerade jetzt die letzte Aufgabe war ja sehr interessant, fand ich. Das mal so zu sehen, dass das... ne also wenn man mal so drüber nachdenkt, den Effekt dann auch ein bisschen zu verstehen. (m24, l. 82)
\end{quote}

\subsubsection*{Tooltip with further information}
As discussed in the previous section, the tooltip could only be realised in the graph that showed the tweet volume over time. This tooltip helped participants to explore the dataset more efficiently:

\begin{quote}
    Screen: Hovert mit dem Mauszeiger über den längsten Balken
    B: Oh, gibt's hier... ne, doch, am 16.6. Ich hatte das eben gar nicht entdeckt, dass es nen Tooltip gibt, wenn man drüber hovert. Sonst hätte ich jetzt hier noch angefangen, die Tage abzuzählen. (m24, ll. 45-47)
\end{quote}

\subsubsection*{Ease of use}
This category contains statements from the participants where they expressed the ease of use of the visualizations.

\begin{quote}
    Ich finde das sehr übersichtlich und eigentlich auch, wenn man sich das einmal angeschaut hat und einmal was wo find ich was, wo kann ich welche Daten aus- und wieder einblenden, schnell und gut benutzbar. Auch wenn man eigentlich nicht so ein Computer-Spezialist ist kann man doch sehr gut damit arbeiten. (w25, l. 78)
\end{quote}

Another participant said that they were able to learn more about how Twitter works than on twitter itself because the tool was easy to use:

\begin{quote}
    Aber prinzipiell ist das ziemlich cool und auch intuitiv. Also man kann ja selbst über diese Twittersuche sich relativ viele Daten selbst ziehen, aber das würdest du ja nicht machen wenn du da nicht selber so direkt mit rumspielen kannst. (w26b, l. 63)
\end{quote}

\subsubsection*{Guide lines in visualization}
Some participants voiced the usefulness of the guideline marking the 0-line in the line graph showing the sentiment (see \ref{fig:sentiment_linechart}):

\begin{quote}
    Ich würde sagen, also die Linie bewegt sich schon im Durchschnitt oberhalb des 0-Wertes hier, aber hier an manchen Tagen reißt sie deutlich nach unten ein. (w26a, l. 66)
\end{quote}

\subsection{Hindering factors}
Hindering factors are factors that hindered the participants in their usage of the tool in one way or another.

\subsubsection*{Hidden filter status}
