\section{Results}
\begin{itemize}
    \item The findings fell into two categories: motivational factors and hindering factors.
\end{itemize}

\subsection{Sample description}
\begin{itemize}
    \item Six participants, aged 23-26, all students
    \item 3 male, 3 female
    \item they were chosen as participants because they have little to no knowledge in big data and data visualization, thus qualifying as \say{laymen}.
    \item 3 read Twitter daily, 1 person at least a few times a week. 1 Participant never reads contents on twitter itself.
    \item 4 people see twitter content at least weekly on other platforms like Instagram or Reddit, two rarely
    \item 1 person reports that they frequently interact with twitter contents. The other 5 rarely or never do this
    \item all of the participants reported that they rarely or never post tweets themselves
\end{itemize}

\subsection{Motivational factors}
Motivational factors were factors that motivated participants to use the dashboard, explore the dataset, and analyse the results. 9 motivational factors were identified in total which will be described in the following, along with anchor examples.

\subsubsection*{Interesting results}
Participants reported that they found the results they gathered from the visualizations interesting. One example is a participant who found it interesting that retweets make up most of the activity on twitter:

\begin{quote}
    Aber schon spannend eigentlich, dass alles... also viele hauptsächlich Retweets sind und nicht eigene Tweets. (m23, l. 18)
\end{quote}

Others found the impact of retweets on the overall sentiment interesting:

\begin{quote}
    Ich fands sehr interessant zu sehen, wie so Retweets die Stimmung beeinflussen. (w25, l. 80)
\end{quote}

Participants also expressed that they liked to have the ability to see how a certain topic was discussed by the German twittersphere:

\begin{quote}
    Aber ja, das ist schon interessant da mal nen Begriff reinzutippen und zu gucken, was passiert. (m24, l. 82)
\end{quote}

Some participants said that it was nice to see their own gut-feeling being backed by data:

\begin{quote}
    Ich finds sehr interessant zu sehen, wie so Retweets die Stimmung beeinflussen. Weil wenn man selbst Twitter benutzt merkt man halt auch selbst [...] dass mich persönlich das aufregt. Und das ist eigentlich sehr interessant gewesen zu sehen, dass das nicht nur mein Empfinden ist, sondern dass sich das generell in Twitter wiederfindet. Dass die Stimmung durch immer-wiederholen des gleichen Themas beeinflusst wird. (w25, l. 80)
\end{quote}

\subsubsection*{Filters can be applied quickly}
During testing, all participants used the toggles to show and hide retweets and neutral tweets multiple times in a row. This was likely done to compare the impact of retweets and neutral tweets on the data set.

\begin{quote}
    B: Und jetzt kann ich ja mal gucken wie sich das ändert, wenn ich die Retweets ausblende

    Screen: blendet die Retweets wiederholt aus und wieder ein (w26a, ll. 73-74)
\end{quote}

This behaviour happened both during the free exploration and the task-solving phase of the interview.

\subsubsection*{Easily understandable visualizations}
Especially the bar chart which showed the tweet volume per day seemed to be easy to read for participants. This led to a quicker understanding of the data set and made it easier for participants to verify their assumptions. This participant, for example, wanted to see if a real-life event she knew of was reflected in the data set during free exploration:

\begin{quote}
    B: Ich probier es mal mit "Stuttgart", denn da gab es ja glaub ich eine der ersten dieser Hygiene-Demos
    Screen: gelbe Balken werden zwischen dem 21. und dem 24. Juni sichtbar
    B: Ah ja interessant, dann war diese Demo bestimmt am 21. Juni. Interessant. (w26b, ll. 6-8)
\end{quote}

Another participant explicitely said that both visualizations were easily understandable, even without further explanation:

\begin{quote}
    Und die Beschriftung der Achsen war auch so eindeutig, dass man... wahrscheinlich hätte ich da nicht einmal den Infotext gebraucht und die Grafiken hätten gereicht. (m26, l. 83)
\end{quote}

\subsubsection*{Supporting explanatory texts}
Participants reported that the explanatory texts that accompanied the visualizations were helpful. They aided them both in the exploration phase:

\begin{quote}
    Ich hab mich gerade kurz gefragt, was neutrale Tweets sind, aber hier unten steht ja die Erklärung dazu. (m26, l. 22)
\end{quote}

as well as while solving the tasks:

\begin{quote}
    I: Wie leicht ist es dir gefallen, die Aufgaben zu beantworten?

    B: Also ich würde sagen ziemlich leicht [...] Mit den Erklärungen fand ich die schon gut zu lösen. (w26b, ll. 68-69)
\end{quote}

\subsubsection*{Encouraging analysis and own thinking}
Another point that participants liked was that the visualizations encouraged them to think about the meaning for themselves. One participant expressed joy that the interactive visualizations helped her answering questions she had on her own:

\begin{quote}
    Ne es gab ja so unendlich viele Corona-Visualisierungen immer. Und ich find das ganz cool, dass man hier sehr offen sich das anschauen kann. Weil sonst geben so Visualisierungen ja ziemlich genau vor, was man überhaupt sehen kann. (w26b, l. 63)
\end{quote}

Another participant liked that the tasks asked them to reflect on what they could read from the visualizations:

\begin{quote}
    Ja gerade jetzt die letzte Aufgabe war ja sehr interessant, fand ich. Das mal so zu sehen, dass das... ne also wenn man mal so drüber nachdenkt, den Effekt dann auch ein bisschen zu verstehen. (m24, l. 82)
\end{quote}

\subsubsection*{Tooltip with further information}
As discussed in the previous section, the tooltip could only be realised in the graph that showed the tweet volume over time. This tooltip helped participants to explore the dataset more efficiently:

\begin{quote}
    Screen: Hovert mit dem Mauszeiger über den längsten Balken
    B: Oh, gibt's hier... ne, doch, am 16.6. Ich hatte das eben gar nicht entdeckt, dass es nen Tooltip gibt, wenn man drüber hovert. Sonst hätte ich jetzt hier noch angefangen, die Tage abzuzählen. (m24, ll. 45-47)
\end{quote}

\subsubsection*{Ease of use}
This category contains statements from the participants where they expressed the ease of use of the visualizations.

\begin{quote}
    Ich finde das sehr übersichtlich und eigentlich auch, wenn man sich das einmal angeschaut hat und einmal was wo find ich was, wo kann ich welche Daten aus- und wieder einblenden, schnell und gut benutzbar. Auch wenn man eigentlich nicht so ein Computer-Spezialist ist kann man doch sehr gut damit arbeiten. (w25, l. 78)
\end{quote}

Another participant said that they were able to learn more about how Twitter works than on twitter itself because the tool was easy to use:

\begin{quote}
    Aber prinzipiell ist das ziemlich cool und auch intuitiv. Also man kann ja selbst über diese Twittersuche sich relativ viele Daten selbst ziehen, aber das würdest du ja nicht machen wenn du da nicht selber so direkt mit rumspielen kannst. (w26b, l. 63)
\end{quote}

\subsubsection*{Guide lines in visualization}
Some participants voiced the usefulness of the guideline marking the 0-line in the line graph showing the sentiment (see figure \ref{fig:sentiment_linechart}):

\begin{quote}
    Ich würde sagen, also die Linie bewegt sich schon im Durchschnitt oberhalb des 0-Wertes hier, aber hier an manchen Tagen reißt sie deutlich nach unten ein. (w26a, l. 66)
\end{quote}

\clearpage
\subsection{Hindering factors}
Hindering factors are factors that hindered the participants in their usage of the tool in one way or another.

\subsubsection*{Hidden filter status}
Participants had trouble recognizing that the search word had an impact on both graphs.

\begin{quote}
    Achso, das gesuchte Wort ist immer noch Covid hier in dem Graph, oder? Ah, okay. (m23, l. 20)
\end{quote}

Other participants were unsure whether the retweet- and neutral tweet filters which were situated between the two visualizations had, in fact, an influence on the second visualization as well.

\begin{quote}
    Screen: Blendet die neutralen Tweets aus und scrollt zum Sentiment-Graph

    B: Eeehm... Moment, hat das hier nen Einfluss drauf?

    Screen: Togglet die neutralen Tweets nochmal an und aus

    B: Ah ja, hat es. (m24, ll. 58-61)
\end{quote}

Participants also reported that they forgot about the status of the filters while they were analysing the visualizations:

\begin{quote}
    Und dann achtet man halt nicht auf solche Häkchen, die man irgendwie setzen kann, die sich dann auf alles auswirken. (w26b, l. 63)
\end{quote}

One participant also said that having to constantly scroll up to the filter status hindered her to effectively solve the tasks:

\begin{quote}
    Also es war eher die Frage, wie interpretiere ich das jetzt, welche Aussage kann ich aus dem, was ich da sehe, treffen. Und dann immer gucken "hab ich die jetzt eingeblendet, hab ich die jetzt ausgeblendet?", also da musste ich schon nen Moment länger drüber nachdenken. (w26a, l. 89)
\end{quote}

\subsubsection*{Missing legend on graphs}
Legends are used in color-coded graphs so that users can map the used colors to their meaning. An example of this can be seen in figure \ref{fig:legend_example}.

\begin{figure}[h!tb]
    \fbox{\includegraphics[width=0.7\linewidth]{images/legend_example.jpg}}
    \caption{Example of a legend. Screenshot from https://observablehq.com/@cahaber/cartesian-legend, highlight by the author}
    \label{fig:legend_example}
\end{figure}

However, the tested tool did not have such a legend. Instead, the meaning of the different colors were explained in explanatory texts of the graphs, as seen in figure %\ref{fig:sentiment_nolegend}.

% TODO: will ich die Grafik drin lassen? Evtl. um Platz zu schinden.
% \begin{figure}[h!tb]
%     \fbox{\includegraphics[width=\linewidth]{images/sentiment_nolegend.jpg}}
%     \caption{The sentiment chart with textual explanation of the colors below it}
%     \label{fig:sentiment_nolegend}
% \end{figure}

This resulted in several issues. One participant did not understand the meaning of the differently colored bars in the bar chart because she did not read the explanatory text thoroughly enough:

\begin{quote}
    B: Meine Frage ist jetzt nur, was bedeutet Orange und was bedeutet blau? 

    Screen: Scrollt etwas runter

    B: Da weiß ich aber nicht, ob das nachher noch kommt. Aber jetzt im ersten Moment hab ich nicht gesehen, was das bedeutet. (w25, ll. 8-10)
\end{quote}

The same participant had trouble recognizing the textually explained color in the sentiment chart.

\begin{quote}
    Screen: Scrollt zum Sentiment-Erklärtext runter

    B: Hä, rot? Ich seh gar keine hellrote Linie, sondern nur hier das Orange und blau.

    I: Ja, die orange Linie ist damit gemeint.

    B: Achso, okay. (w25, ll. 24-27)
\end{quote}

Another participant got confused by the textual explanation of the different parts of the chart. In this case, the word \say{blue} could have meant two different lines, one dark blue and one light blue.

\begin{quote}
    "Die Linie zeigt die durchschnittliche Stimmung..." Aber welche Linie ist denn jetzt DIE Linie? Dann gibt es eine blaue und eine hellblaue. (m24, l. 65)
\end{quote}

\subsubsection*{Layout of charts and texts}
This category contains problems that arose during testing that stem from the layout of charts and texts. One participant did not see the explanatory text for the sentiment graph as it was below the graph itself:

\begin{quote}
    B: Okay, ich glaub ich hab noch nicht ganz verstanden, was mir der Graph sagen will. Also, Sentiment ist quasi die Wertung?

    I: weiter unten steht auch noch mehr Erklärtext dazu.

    B: Okay. Aaah okay, das ist der Text hier unten (m23, ll. 22-24)
\end{quote}

Another participant seemed to be confused by the sentiment chart. After he found the explanatory text below the chart, he could continue with the task.

\begin{quote}
    Screen: Scrollt wieder zum Sentiment-Graphen runter
    
    B: Ach aber auch hier kein besonders überraschendes Sentiment. Warte, was... was tut denn hier die blaue?

    Screen: Fährt mit dem Cursor über den Erklärtext unter dem Graphen
    
    B: Ah, okay. Joar, okay. Dann... vielleicht doch eher. (m24, ll. 28-31) 
\end{quote}

One participant explicitly stated that he would have preferred to have the explanation texts \emph{above} the charts.

\begin{quote}
    Ich glaub das einzige, was ich gemacht hätte, wäre, den Infotext über die Grafik zu packen. Das wär für mich praktischer gewesen, weil dann lese ich erst, was ich sehe, und sehe dann erst die Grafik. Wobei ich jetzt gerade auch nen relativ kleinen Bildschirm habe. (m26, l. 75)
\end{quote}

However, one participant who studies to become a teacher said that she liked the layout from a pedagogical perspective.

\begin{quote}
    I: Das heißt, für deinen Lesefluss und das Verständnis wäre es besser gewesen, wenn du den Erklärtext über der Darstellung hast, die erklärt wird?

    B: Das ist ne gute Frage, also ich glaube aus einer Verständnistheoretischen Perspektive ist es wahrscheinlich schlauer, wenn man sich erst die Frage stellt, bisschen irritiert ist und dann die Antwort findet und dann das versteht. Wenn man es erst liest und nicht genau weiß, worauf sich das jetzt bezieht, muss man den Text im Zweifelsfall zwei Mal lesen. Weil dann liest du das halt, denkst "hä?", dann schaust du dir das an und merkst dann, dass du Informationen dazu bekommen hast. Und dann liest du dir das nochmal durch. (w26a, ll. 80-81)
\end{quote}

\subsubsection*{Word search is a filter}
This category contains statements where the participants stumbled over the limitations of the word filter. The word filter filters for exact matches of the entered term in the database. This is a difference to search engines like Google or Duck Duck Go: these engines use a so-called \emph{fuzzy search}, which means that results don't only contain the exact search term, but also similar terms.

One participant entered a too specific keyword, but could quickly recover from this:

\begin{quote}
    Screen: Gibt "Dr. Drosten" in den Suchfilter ein

    B: Dann würd ich eben erst mal Dr. Drosten in diese Suchleiste hier oben eingeben

    Screen: wenige Ergebnisse werden gezeigt

    B: Wobei vielleicht geb ich besser nur "Drosten" ein weil den Doktor... filtern die Leute ja wahrscheinlich raus. (w26b, ll. 37-40)
\end{quote}

Another participant made a spelling mistake when filtering for a word, which she did not recognize until she was made aware of it:

\begin{quote}
    Screen: Gibt im Wortfilter "Tonnies" ein, es werden nur sehr wenige Ergebnisse angezeigt

    B: Hm, jetzt seh ich hier gar nix

    I: Du hast T\"onnies falsch geschrieben. Mit O statt mit \"O.
    
    B: Aah, I see. 

    Screen: Gibt T\"onnies ein

    B: Okay, und verändert sich jetzt was? Ah ja, okay!  (w26a, ll. 10-15)
\end{quote}

Without the author's intervention, this participant likely would not have been able to recover from her mistake, thinking that \emph{Tönnies} did not appear in the set of collected tweets.

\subsubsection*{Missing real-world context}
While the participants were able to identify days with high activity for a specific topic, the dashboard did not offer more information about what happened on that day. Some participants said that they could have used this information to interpret the findings better.

\begin{quote}
    B: Ganz witzig, am 5. und am 19. Juli gibts hier so... wie nennt man das denn, kein Peak sondern das Gegenteil  eines Peaks wo es dann so krass runtergeht. Das wäre interessant zu wissen, was an diesen Tagen war, ob die Leute da wohl schlechter drauf waren in Bezug auf Corona.  (w26b, l. 18)
\end{quote}

\subsubsection*{Fetching status was unclear}
When the word filter is changed, new data is fetched from the database. This means that it takes about five seconds between typing in a new search word and the results for this word being displayed. Participants felt unsure about whether their input had already been processed or if they have to wait a bit longer, as the fetching status was not shown in the interface.

\begin{quote}
    B: Da geb ich oben in diesem Suchbalken was ein. Da geb ich jetzt Dr. Drosten ein

    Screen: Die Grafik verändert sich, zeigt aber nahezu keine Ergebnisse. 

    B: Und... drücke Enter? Hat sich jetzt schon was getan?

    I: Ja, hat es.

    B: Ach, das hab ich gar nicht gemerkt. (w25, ll. 36-40)
\end{quote}

\subsubsection*{Unprecise explanations}
This category contains statements from participants where an explanation for certain elements was given, but the explanation was not satisfactory. This contains unprecise wording or missing further information.

\begin{quote}
    Sentiment ist noch so... ich weiß noch nicht so ganz, was das heißt. Also klar, was das irgendwie bedeuten soll, aber... wie sieht jetzt zum Beispiel ein positiver Tweet aus oder ein negativer. (m23, l. 73)
\end{quote}

\subsubsection*{Tooltip was not found}
As discussed before, the bar chart had tooltips which showed more detailed information about the bar the user hovers on. However, these tooltips were not always found. One participant, for example, counted the bars while solving task 2 (finding the day where most tweets were sent about Dr. Drosten):

\begin{quote}
    B: Aha! Und dann würde ich behaupten... das ist der 24., 25., ich würde sagen das ist der 27. wobei das relativ close ist mit dem... 22., 23., 24. 

    Screen: Das Popup erscheint zwar, sie zählt die Tage trotzdem noch an der X-Achse ab (w26a, ll. 60-61)
\end{quote}

