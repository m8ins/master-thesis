\section{Results}
\begin{itemize}
    \item The findings fell into two categories: motivational factors and hindering factors.
\end{itemize}

\subsection{Sample description}
\begin{itemize}
    \item Six participants, aged 23-26, all students
    \item 3 male, 3 female
    \item they were chosen as participants because they have little to no knowledge in big data and data visualization, thus qualifying as \say{laymen}.
    \item 3 read Twitter daily, 1 person at least a few times a week. 1 Participant never reads contents on twitter itself.
    \item 4 people see twitter content at least weekly on other platforms like Instagram or Reddit, two rarely
    \item 1 person reports that they frequently interact with twitter contents. The other 5 rarely or never do this
    \item all of the participants reported that they rarely or never post tweets themselves
\end{itemize}

\subsection{Motivational factors}
\begin{itemize}
    \item \textbf{Interesting results}: Participants reported that they found the results interesting. One example is a participant who found it interesting that retweets make up most of the activity on twitter:
    \begin{quote}
        Aber schon spannend eigentlich, dass alles... also viele hauptsächlich Retweets sind und nicht eigene Tweets. (m23, l. 18)
    \end{quote}
    Others found the impact of retweets on the overall sentiment interesting:
    \begin{quote}
        Ich fands sehr interessant zu sehen, wie so Retweets die Stimmung beeinflussen. (w25, l. 80)
    \end{quote}
\end{itemize}
