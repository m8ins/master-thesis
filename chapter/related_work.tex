This part reviews and compiles work that has already been done on this.

\subsection{Twitter as microblogging platform}
The microblogging platform \emph{Twitter} allows users to post messages that are up to 240 characters long, so-called \emph{Tweets}. Users can follow other users; each user's personal timeline is then populated by tweets of people they follow (\cite{thimm_twitter_2012}). Accounts can also be verified; this \say{lets people know that an account of public interest is authentic} (\cite{twitter_inc_about_nodate-1})

Seeing that Twitter users share their opinions on many different topics openly, the platform offers a good data source for researchers to assess the public's opinion (\cite{pak2010twitter}, \cite{pfaffenberger2016twitter}, \cite{broniatowski2014twitter}). Twitter has also a very diverse audience: Politicians and journalists are using the platform the same way as celebrities or regular users (\cite{pak2010twitter}), which lets researchers collect data from different social groups and groups with diverse interests.

Furthermore, Twitter offers an API which lets developers access the tweets programatically, instead of scraping the website for its contents (\cite{twitter_inc_about_nodate}). The API allows easy access to tweets and provides functionality to filter tweets by keywords, hashtags, language or geographic regions (\cite{bello2017detecting}).

\subsection{Twitter and public opinion}
When Osama bin Laden was killed, the information was first spread via Twitter before it hit major news outlets (\cite{hu2012breaking}). 
Even though the kill had not yet been confirmed at that time, a certainty analysis of the tweets sent between the first rumours and the official confirmation showed that many Twitter users were convinced that the rumours were true; this makes the bin Laden killing one of the earliest cases of Twitter being faster than traditional media, but equally accepted as a trustworthy source(\cite[2751]{hu2012breaking}).

A study examining the social media-usage around the protests of the 2010 G20 summit shows that Twitter was a suitable platform for crowd-sourced journalism as opposed to traditional mainstream journalism (\cite{poell2012twitter}). The study analysed the usage of three different social media-platforms during the protests: Twitter, YouTube and Flickr. Of the three Of the three platforms the researchers examined, Twitter was used by more people, and the retweeting-feature gave more silent users the possibility to weigh in without having to create own content (cf. \cite[709]{poell2012twitter}).