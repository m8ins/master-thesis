This part reviews and compiles work that has already been done on this.

\subsection{Twitter as microblogging platform}
The microblogging platform \emph{Twitter} allows users to post messages that are up to 280 characters long, so-called \emph{Tweets}. Users can follow other users; each user's  timeline is then populated by tweets of people they follow (\cite{thimm_twitter_2012}). Accounts can also be verified; this \say{lets people know that an account of public interest is authentic} (\cite{twitter_inc_about_nodate-1}). 

Seeing that Twitter users share their opinions on many different topics openly, the platform offers a good data source for researchers to assess the public's opinion (\cite{pak2010twitter}, \cite{pfaffenberger2016twitter}, \cite{broniatowski2014twitter}). Twitter also has a very diverse audience: Politicians and journalists are using the platform the same way as celebrities or regular users (\cite{pak2010twitter}), which lets researchers collect data from different social groups and groups with diverse interests. And opposed to other social networks, like Facebook, Tweets can be viewed by all users instead of only those in their network (\cite{park_does_2013}).

Twitter is also a medium that is widely used by the 'ordinary public'. A 2011 study conducted in South Korea shows that the majority of the most popular 1\% of tweets were sent by ordinary people instead of celebrities (\cite{chang2011structure}, as cited in \cite{park_does_2013}).

Because of this openness, some scholars say that Twitter should be documented and archived as a historic record (\cite{risse2014documenting}). Twitter documents today's society in great detail -- and, more importantly, individuals document their own experiences. This is a stark contrast to (historic) documentation in the past, where no direct documentation happened, but rather compiled reports of sorts. \citeauthor{risse2014documenting} argue that not only the text of a tweet is important, but also the context, like who sent the tweet and at what time. They also argue that further natural language processing is needed so that an archive could be searched for specific topics. This necessary natural language processing is no easy task on Twitter, however: due to the 140-character-limit (which was doubled to 280 in recent years), a lot of vernaculars and abbreviations are used, which can be difficult for natural language processors to process. Nevertheless, the authors claim that \say{Twitter can be seen as a comprehensive documentation of society} \cite[9]{risse2014documenting}, and saying that this documentation can be of immense historical value for future generations.

Furthermore, Twitter offers an API that lets developers access the tweets programmatically, instead of scraping the website for its contents (\cite{twitter_inc_about_nodate}). The API allows easy access to tweets and provides functionality to filter tweets by keywords, hashtags, language, or geographic regions (\cite{bello2017detecting}).

\subsection{Twitter and public opinion}  % Maybe "Twitter and crowd journalism" or something?
When Osama bin Laden was killed, the information was first spread via Twitter before it hit major news outlets (\cite{hu2012breaking}). 
Even though the kill had not yet been confirmed at that time, a certainty analysis of the tweets sent between the first rumors and the official confirmation showed that many Twitter users were convinced that the rumors were true; this makes the bin Laden killing one of the earliest cases of Twitter being faster than traditional media, but equally accepted as a trustworthy source(\cite[2751]{hu2012breaking}).

A study examining the social media-usage around the protests of the 2010 G20 summit in Toronto shows that Twitter was a suitable platform for crowd-sourced journalism\cite{poell2012twitter}. Crowd-sourced journalism, or alternative journalism, positions itself as the opposite of mainstream journalism. While mainstream coverage of protests often focus on the events themselves, rather than the underlying issues people are protesting against, crowd journalism has the chance to shine a light on why people are protesting \cite[698]{poell2012twitter}. The study analyzed the usage of three different social media platforms during the protests: Twitter, YouTube, and Flickr. Of the three platforms the researchers examined, Twitter was used by more people, and the retweeting-feature gave more silent users the possibility to weigh in without having to create their  content (cf. \cite[709]{poell2012twitter}). However, the authors concluded that the content published on social media-platforms was focused on sensationalist messages. This means that instead of focusing on the meaning of the protests, as intended in crowd-sourced journalism, the focus lay again on sensationalist pictures.

\subsection{Filter Bubble in social media}
While social media can connect people all over the world, scientists believe that their recommender algorithms are designed in a way that amplifies user's own opinions (\cite{pariser2011filter}). This leads to worries that discussions online become less diverse. Rather than being exposed to a plethora of viewpoints and arguments, social media platforms and search engines decrease information diversity to improve engagement on their websites (\cite{bozdag_breaking_2015}). One study found that especially in political issues Twitter users tend to communicate mainly with users who share their political views (\cite{barbera_tweeting_2015}).

How big the effect of the filter bubble is remains unclear (\cite{bruns_echo_2017}). A study examining the Australian twittersphere has found that while there are some clusters of Twitter users, \say{those connections have not been made to the exclusion of all others} (\cite[9]{bruns_echo_2017}), saying that from a network-centric point of view filter bubbles don't really exist. However, this study fails to incorporate Twitter's default news feed which shows algorithmically-picked posts first rather than a chronological timeline of all activity in a network.

One survey conducted in the UK showed that politically interested users try to avoid echo chambers (\cite{dubois_echo_2018}). 

%It appears as if Twitter is, on the one hand, a grassroots medium, allowing the broader public to engage in discussion and form an opinion


\subsection{Natural Language Analysis on Twitter}

Different kinds of research has been done to use Twitter data as predictor for social behaviour. \citeauthor{bahk2016publicly} developed a prototype for a dashboard that helps monitoring the populations' sentiment toward vaccinations. The so-called 'antivaxxers' pose a real thread to the wider population, because vaccines only work effectively when a lot of people are vaccinated. Thus, it is interesting to know when and where negative sentiment towards vaccinations arise to formulate counter measures, like targeted information campagins. The authors searched Twitter for vaccine-related messages and coded them using sentiment analysis. The data is presented on a dashboard, where the negative sentiment per country is shown, along with further filters. According to the authors, \say{[the dashboard] can be used to detect early signals in shifting conversations about vaccination} (\cite[343]{bahk2016publicly}). 

% Idee dieser Subsection: Erklären, was alles dadurch erreicht werden kann, dass NLA auf Tweets angewandt wird. Beispiele: Vaccination Dashboard, rausfinden dass anti-vaccine tweets häufiger retweetet werden als pro-vaccine tweets

\subsection{Why is Datavis important?}

%TODO: Schöneren Subsection-Titel finden

In the last couple of years we generated more data than in the combined history of human kind (\cite{helbing2019will})---and with the Internet of Things on the rise, experts assume that the amount of data will double every 12 hours.

The more data we collect, %and the more ubiquitious data-driven decision making becomes (\cite{brynjolfsson_strength_2011}),
the more important it becomes to understand big data (\cite{borner_data_2019}). As human perception is fine-tuned to recognise underlying patterns, trends and outliers, \emph{Data Visualization} is one of the most common tools to explore and communicate big data sets (\cite{heer_tour_2010}), as it helps to make sense of big data in a more intuitive way than reading data base tables or big charts (\cite{donalek_immersive_2014}).

However, studies have shown that laymen don't have the required knowledge to accurately read complex visualisations (\cite{borner_investigating_2016}). %This means that data visualizations have to be constructed carefully and assume almost no previous knowledge when it comes to explanatory texts.
Even though our visual system is trained to spot trends and outliers, data visualisation is not inherently self-explanatory. Our ability to extract information from visualisations to answer questions based on data sets is called \emph{Data Literacy} (\cite{boy_principled_2014}). Improving this literacy allows people to not only improve their communication and collaboration on data sets, but also their understanding of the world as shown by big data \cite{borner_data_2019}). 
%A good data visualization does not provide definite answers in itself; rather, it should support the users in the search for results (\cite{light2001portable}).
