\section{Conclusion}

- the impact of social media on society, politics, and well-being is not likely to decrease in future years. Meanwhile, users of social media cannot get an objective view of the network, as they mostly see contents that are chosen by an engagement-driving algorithm. This could pose a real threat to society and a rise of populism (Groshek2017).

- this study examined ways to automatically collect, process, and store large amounts of Twitter data and make such a dataset accessible to laymen. This could give them the tools to examine how specific topics are discussed on social media, making this less opaque.

- The visualizations were tested in user studies to see if laymen can gain meaningful insights from the visualizations.

- the user study showed that laymen could gain insight from these visualizations while analyzing further improvements that future studies could focus on.