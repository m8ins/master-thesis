\section{Conclusion}
The impact of social media on society, politics, and personal well-being is not likely to decrease in the future. This relationship between social media and its users is an asymmetrical one, however: Users cannot gain insight into how social media works, how the algorithms select contents, and most importantly  how social media might skew one's perception of reality. Already today, this poses a real threat to society as the engagement-driven recommender algorithms support a rise of populism (\cite{groshekHelpingPopulismWin2017}).

This study examined ways to automatically collect, process, and store large amounts of Twitter data. Then, using visualizations, this data set was made accessible to laymen. Using such a tool to examine how specific topics are discussed on Twitter makes it easier for users to understand what is going on in the world. Tools like the one presented in this study can even be seen as the future of journalism: not a one-way street of pre-analyzed data, but rather giving readers \say{the tools to hold institutions in your life accountable for their choices.} (\cite{angwinMakingPrivacyPersonal2020}).

The visualizations were tested in user studies to see if laymen can indeed gain meaningful insights from the visualizations. While some potential for improvement was found, the results generally hint at laymen indeed being capable of reading visualizations, as long as some rules and guidelines are followed.


%- the impact of social media on society, politics, and well-being is not likely to decrease in future years. Meanwhile, users of social media cannot get an objective view of the network, as they mostly see contents that are chosen by an engagement-driving algorithm. This could pose a real threat to society and a rise of populism (Groshek2017).

%- this study examined ways to automatically collect, process, and store large amounts of Twitter data and make such a dataset accessible to laymen. This could give them the tools to examine how specific topics are discussed on social media, making this less opaque.

%- The visualizations were tested in user studies to see if laymen can gain meaningful insights from the visualizations.

%- the user study showed that laymen could gain insight from these visualizations while analyzing further improvements that future studies could focus on.